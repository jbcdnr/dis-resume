\section{Frequent Itemsets}

\subsection{Introduction to Data Analysis}

We gather more and more data (2x more each year) and most of it is useless. Recently the rate of data growth started exceeding the rate of storage growth.
Interesting information can be gathered from this large amount of data (data analytics, data mining).
This can be done in two ways: Data management (this is mainstream Big Data, i.e. ways of storing, indexing and retrieving data) and Data analytics.
Data analytics is the "Analysis of large datasets to find relationships and to summarize the data in novel ways that are both understandable and useful." A classical example is the shopping basket analysis (people who buy diapers also buy beer on Friday afternoons).

\subsection{Components of Data Analytics}
Data analytics can be based on:
\begin{itemize}
\item Local properties (patterns applying to part of the data)
\item A global descriptive model (structure description)
\item A global predictive model (function on the data)
\item Exploratory data analysis (interactive tools and visualisation)
\end{itemize}

Four main components:
\begin{enumerate}
\item Pattern structure and model representation
\item Scoring function
\item Optimization and search
\item Data management
\end{enumerate}

Data analytics is part of a larger system which:
\begin{itemize}
\item Collects and integrates potentially heterogeneous data into data warehouses
\item Selects subsets of the data to perform specific data mining tasks (those are called data-marts)
\item Performs actual data mining on the data-marts
\item Assesses the results of data mining by evaluating the usefulness of the mined patterns
\end{itemize}

\subsection{Pattern structure}
An association rule is a statement of the form

$Body \implies Head [support, confidence]$

where Body and Head are predicates.
A rule can be single-dimensional (body depends on only one predicate) or multi-dimensional (body depends on multiple predicates). Multi-dimensional rules can be transformed into single-dimensional rules by creating a new predicate englobing all the others.
Example: (x is 19-25 y.o. and x buys chips) becomes (x is a customer, who is 19-25 and buys chips)

\subsection{Scoring function}
Support: $P[\{Body, Head\}]$ (The rule can be applied often)
Confidence: $P[\{Head\} | \{Body\}]$ (The rule means something)

\subsection{Optimization and search}

\subsubsection{Definitions}
Given a set of items $I$, an itemset is a subset of $I$.
A transaction is a tuple $(id, T)$ where $T$ is an itemset.
We have a database of transactions $D$.
Given these definitions, an association rule $A \implies B [s, c]$ is characterized as follows:
\begin{itemize}
\item A and B are disjoint itemsets appearing in at least one transaction.
\item support $s = p(A \cup B) = \frac{count(A \cup B)}{|D|}$
\item confidence $c = p(B|A) = \frac{s(A \cup B)}{s(A)}$
\end{itemize}

Now given a database of transactions, we want to get all rules $A \implies B [s,c]$ such that $s > s_{min}$ and $c > c_{min}$.

\subsubsection{Frequent itemsets method - The Apriori algorithm}
This method consists in finding $J \subseteq I$ such that $p(J) > s_{min}$, i.e. itemsets with sufficient support.
This is based on the assumption that $A \implies B$ can only be an association rule if $A \cup B$ is a frequent itemset.
Apriori property: if an itemset is frequent, then any subset of this itemset is frequent too (it appears at least as often as the enclosing itemset).
Given that, we can find frequent itemsets by increasing cardinality.

\paragraph{JOIN step}
Given all frequent itemsets of size $k-1$, The JOIN step consists of finding potential frequent itemsets of size $k$.
To do so, we create a $k$-itemset out of every pair of $(k-1)$-itemsets that differ only by one element, by taking their union.
The efficient way of doing this is as follows:
\begin{itemize}
\item Sort each itemset (by some ordering on the items)
\item Put together the itemsets which have their $(k-2)$ first elements in common.
\item For each of these sets of itemsets, create $k$-itemsets consisting of the common $(k-2)$ first items, and every possible pair of the last elements of the $(k-1)$-itemsets.
\end{itemize}
This method minimizes the number of generated potential frequent itemsets, while still being conservative.

\paragraph{PRUNE step}
Given the candidate frequent itemsets computed at the JOIN step, the PRUNE step removes those which contain a non-frequent $(k-1)$-itemset.
Indeed, as all subsets of a frequent itemset are frequent, containing an itemset that is not frequent is sufficient to decide that the itemset is not frequent.

\paragraph{Final step}
Given the surviving itemsets from the PRUNE step, the final step consists of counting the number of occurences of each of these itemsets in the database.
This is an expensive step (full access to the DB), but its size is greatly reduced by the JOIN and PRUNE steps.
The Apriori algorithm is therefore the succession of the JOIN, PRUNE and Final steps on increasing itemset sizes (from $k = 1$) until the resulting set is empty. The set of frequent itemsets is therefore the union of the sets computed at each step.

\subsubsection{Pertinent rules}
Once the frequent itemsets are computed, we compute the pertinent rules as follows:
\begin{itemize}
\item Given a frequent itemset $J$, a candidate rule is $A \implies J/A$, for all $A \subseteq J$. (Recall that an association rule $A \implies B$ is such that $A \cup B$ is a frequent itemset and $A$ and $B$ are disjoint).
\item The rule has enough support (frequent itemsets were defined to satisfy this), it also needs good confidence. Confidence is $\frac{s(J/A \cup A)}{s(A)} = \frac{s(J)}{s(A)}$.
\item As $J$ is a frequent itemset, so is $A$ and both support counts are available from the Apriori algorithm, so the computation is straightforward
\end{itemize}

\subsubsection{Interest of association rules}
Some rules can have high confidence if their head is very frequent.
We define the Interest of a rule $I \implies j$ by $c(I \implies j) - P[j]$, which is the difference between the confidence of the rule and the fraction of transactions that contain $j$.
Intuitively, this means that the rule tells us something interesting, that we couldn't have inferred from the frequency of the head.

\subsubsection{Quantitative attributes}
For now the search was applied on a finite item space. When applying it to continuous and/or infinite item spaces, one needs to discretize the data before applying the Apriori algoritm. This can be done in two ways:
\begin{itemize}
\item Static discretization: put the values in predefined bins
\item Dynamic discretization: compute the bins based on the distribution of the data
\end{itemize}

\subsection{Data management}
The following techniques can be used to improve the Apriori algorithm on large datasets:
\begin{itemize}
\item Transaction reduction: when looking for frequent itemsets, a transaction that doesn't contain any frequent $k$-itemset can be omitted in scans for bigger itemsets.
\item Partitioning: a frequent itemset must be frequent on at least one partition of the data. We can therefore mine on partitions, and compute the exact frequencies only once.
\item Sampling: we can mine on a sampled subset of the DB (with a lower minimal support), and compute the exact frequencies only once against the whole DB.
\end{itemize}

\paragraph{The SON algorithm}
This implements partitioning. In a first pass, we read the whole transaction database chunk by chunk, such that every chunk fits in memory. An itemset is a candidate if it is frequent in any of the chunks.
On a second pass, we count all the candidate itemsets and determine which are frequent in the entire set.
The SON algorithm can be efficiently computed using MapReduce:
\begin{itemize}
\item Compute frequent itemsets at each node
\item Distribute candidates to all nodes
\item Accumulate the counts of all candidates
\end{itemize}